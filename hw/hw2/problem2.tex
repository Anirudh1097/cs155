\begin{prob}  (Stealth Port Scanning) 
\end{prob}

\noindent Recall that the IP packet header contains a 16-bit identification field that is used for assembling packet fragments. IP mandates that the identification field be unique for each packet for a given (SourceIP,DestIP) pair. A common method for implementing the identification field is to maintain a single counter that is incremented by one for every packet sent. The current value of the counter is embedded in each outgoing packet. Since this counter is used for all connections to the host we say that the host implements a global identification field.\\

\noindent \textit{a) Suppose a host P (whom we'll call the Patsy for reasons that will become clear later) implements a global identification field. Suppose further that P responds to ICMP ping requests. You control some other host A. How can you test if P sent a packet to anyone (other than A) within a certain one minute window? You are allowed to send your own packets to P.}\\

\noindent\textbf{Step 1: } Send an ICMP echo request from $A$ to $P$ at time $t=0$. $P$ will respond with an ICMP echo response packet and $A$ will remember the counter value $X$ in the identification field of the response.\\
\noindent\textbf{Step 2: } At time $t=1min$ send another ICMP echo request from $A$ to $P$. $P$ will respond with an ICMP echo response and $A$ will remember the counter value $Y$ in the identification field of the response.\\
\noindent\textbf{Step 3: } If $Y=X+1$ then we conclude that $P$ has not been talking to anyone else since it has incremented its global identification field just once during that minute for the 2nd ping response it sent us\footnote{we ignore the possibility of $Y$ wrapping around to $X+1$ since it would imply $P$ sent around $2^{16}$ packets in a minute which highly unlikely}. If $Y>X+1$ then we conclude that $P$ has sent a packet to someone other than $A$ during that minute since it incremented its counter once for the 2nd ping response to us and at least once for a packet it sent to someone else.\\
 

\noindent \textit{b) Your goal now is to test whether a victim host V is running a server that accepts connections to port n (that is, test if V is listening on port n). You wish to hide the identity of your machine A and therefore A cannot directly send a packet to V, unless that packet contains a spoofed source IP address. Explain how to use the patsy host P to test if V accepts connections to port n.\\
Hint: Recall the following facts about TCP:\\
-- A host that receives a SYN packet to an open port n sends back a SYN/ACK response to the source IP.\\
-- A host that receives a SYN packet to a closed port n sends back a RST packet to the source IP.
A host that receives a SYN/ACK packet that it is not expecting sends back a RST packet to the source IP.\\
-- A host that receives a RST packet sends back no response. }\\

\noindent\textbf{Step 1: } Send an ICMP echo request from $A$ to $P$ at time $t=0$. $P$ will respond with an ICMP echo response packet and $A$ will remember the counter value $X$ in the identification field of the response.\\
\noindent\textbf{Step 2: } Immediately after that at time $t=0$, have $A$ send a TCP SYN packet to $V$ with spoofed source address equal to $P$'s IP address and destination port equal to $n$.\\

When $V$ gets the SYN packet two things can happen: 

\textbf{a)} if $V$ is not listening on port $n$ then it will send a TCP RST packet destined to $P$. When $P$ receives this RST packet\\\indent from $V$ it will not respond with any packets. 

\textbf{b)} if $V$ is listening on port $n$ it will respond to the SYN packet with a SYN/ACK packet destined to $P$. When $P$ receives\\\indent the SYN/ACK packet from $V$ it will send back to $V$ a RST packet since it was not expecting a SYN/ACK and finally $V$\\\indent will not respond to the RST. \\

The differences between these two cases is that in \textbf{a)} $P$ does not send any packets to $V$ and in \textbf{b)} $P$ sends 1 packet to $V$.\\\indent Hence, now we can use our technique from part a) to detect which case holds. \\

\noindent\textbf{Step 3: } At time $t=1min$ have $A$ send another ICMP echo request to $P$. $P$ will respond with an ICMP echo response and $A$ will remember the counter value $Y$ in the identification field of the response.\\
\noindent\textbf{Step 4: } If $Y=X+1$ then we conclude that $P$ did not send a RST packet to $V$, hence did not receive a SYN/ACK from $V$ and hence that $V$ is not listening on port $n$. If $Y>X+1$ then we \textit{infer} that $P$ sent a RST packet to $V$, which implies $V$ sent a SYN/ACK to $P$, which in turn implies $V$ is listening on port $n$.\\

Note that in this analysis we assume two things: 1) in a 1 minute interval $P$ will have received the SYN/ACK or RST packet that $V$ will send. This seems like a reasonable assumption. 2) in order to infer from $Y>X+1$ that $P$ sent a RST packet to $V$ we assume that at least one of the $Y-X-1$ packets sent from $P$ to hosts other than $A$ during that 1 minute was a RST packet to $V$. This is a reasonable assumption if we know $P$ does not get much traffic. In reality we would probably want to decrease the interval to decrease the chances of getting spurious traffic into $P$ and we would likely try this many times in order to filter out any spurious traffic into $P$.
